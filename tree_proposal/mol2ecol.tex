\documentclass[12pt]{article}
\usepackage[margin=1in]{geometry}
\geometry{letterpaper}
\usepackage{graphicx}
\usepackage{setspace}
\usepackage{amssymb}
\usepackage{amsmath}
\usepackage{epstopdf}
\usepackage[numbers]{natbib}
\usepackage{authblk}

\title{Linking evolutionary and ecological theory illuminates
  non-stationary biodiversity}

\author[1]{A. J. Rominger}
\author[2]{ J. Y. Lim}
\author[3]{E. E. Armstrong}
\author[1]{H. Krehenwinkel}
\author[1]{R. G. Gillespie}
\author[1]{J. Harte}
\author[4]{M. J. Hickerson}
\affil[1]{Department of Environmental Science, Policy and Management,
  University of California, Berkeley}
\affil[2]{Department of Integrative Biology, University of California, Berkeley}
\affil[3]{Biology Department, University of Hawaii, Hilo}
\affil[4]{Biology Department, City College of New York}
\renewcommand\Affilfont{\footnotesize}

\date{}

\begin{document}
\maketitle

Whether or not biodiversity dynamics are governed by stable equilibria
remains an unsolved question in ecology and evolution
\citep{quental2013, rabosky2015amNat, harmon2015amNat}. An ecological
steady state exists if changes in biodiversity occur slowly and in
sync with environmental changes \citep{harteNewman}. The existence (or
non-existence) of such steady states has wide ranging implications,
for example, whether conservation should focus on conventional
preservationist paradigms or adaptive management \citep{levin1999}.
Whether biodiversity rapidly and consistently tends toward a steady
state also determines how species and communities will respond to
global environmental change \citep{barnosky2012}. Evolutionary
genetics \citep{nielsen2005, nei2010} and macroecology
\citep{brown1995} represent two primary lenses through which we view
biodiversity dynamics. Population genetics and phylogenetics provide
an integrated view of changes in population demography and lineage
origination over scales of generations to geological epochs but cannot
be used to directly infer the ecological composition of any given time
period \citep{quental2010}. Conversely, macroecology, and its
constituent theories, provides a static description of the commonness
and rarity of species co-occurring in the same landscape
\citep{hubbell2001, harteNewman}.  With advances in molecular methods
\citep{mccormack2013} allowing genetic data to be gathered across
entire communities \citep{gibson2014}, the field is
ready to not only combine the theoretical insights of genetics and
ecology, but also test predictions with real data.

% Theory development in both molecular evolution and macroecology are
% based on the mathematics of long-term stationarity \citep{kimura,
%   hubbell2001}, thus assuming that the systems they
% describe are in steady state. This feature of all theories derives
% both from the mathematical simplicity of stationarity and the unusual,
% albeit short-term, global stability observed during the formative
% period of the study of ecology. As a result the study of biodiversity
% as non-stationary phenomenon has been limited by a lack of conceptual
% tools for describing alternate, non-steady states and a lack of
% hypotheses to test for the causes and consequences of
% non-stationarity.

Neutral or null models in both population genetics/phylogenetics and
ecology seek to provide idealized descriptions of systems free of
additional complicating mechanisms \citep{nei2010, hubbell2001,
  harteNewman}. Departures from these idealizations can be seen as
evidence for violations of the core theoretical assumptions of the
model, such as selective neutrality and constant population size in
the case of population genetics\citep{nielsen2005, nei2010}, or biotic
interactions and habitat filtering in the case of ecology
\citep{borer2014, mittelbach2015}.  As such, departures from mechanistically
idealized theories can be more informative than conformation to those
theories.  Combining the insights from both theoretical perspectives
can shed light on the mechanistic reasons for departures from steady
state.

We posit that two primary classes of non-steady state exist that can
be better understood by combining ecological theory with comparative
population and phylogenetic insights.  The first occurs when a
biological assemblage is undergoing succession following disturbance
or formation of new habitat; in this case populations of most species
in the community and species composition itself will be in flux due to
the stochasticity of immigration and small population sizes.  In such
a situation the assemblage may be expected to eventually converge on a
steady state \citep{simberloff1970}. Recovery from disturbance
\citep{simberloff1970}, range expansion following climate change
\citep{blois2010} and primary succession
\citep{shipley2006} are all potential examples of such non-steady
state. The second case occurs when novel mechanisms actively drive an
assemblage away from steady state; such mechanisms could include
escalatory species interactions or rapid diversification and
adaptation in the face of newfound selective pressures
\citep{rominger2015GEB}. In both cases idealized ecological theory
should fail to predict the static biodiversity patterns of the system
and departures from population genetic theory should indicate what
demographic dynamics are associated with the failure of ecological
theory.
% This perspective immediately begs the question of how many
% species must have population-level dynamics out of molecular steady
% state for there to be a detectable signal of the entire community
% deviating from ecological steady state. We will explore this question
% using simulation to not only evaluate how many species deviating from
% molecular steady state are needed, but also whether the population
% dynamics of common or rare species are more influential to deviation
% from ecological theory.

% Understanding the ecological basis for different evolutionary outcomes
% could prove more challenging because the ecological conditions of the
% present do not necessarily represent the conditions of the past (when
% selection, divergence or diversification began). However, combining
% ecological and genetic theory with time perspectives provided by the
% fossil record \citep{wagner2006, davidsonErwin2006} or chronosequence
% \citep{rominger2015GEB} may provide a solution.  These study systems
% can provide snap shots of both ecological patterns (species
% composition and interactions) as well as their evolution through
% phylogenetic, landscape genetics on chronosequences and ancient DNA
% analysis.

In our manuscript for {\it Trends in Ecology and Evolution} we will
explore the specific hypotheses that can be tested by combining
ecological and population genetic theory. We will use simulated
results to demonstrate how unique evolutionary/demographic scenarios
leave distinct signatures in both genetic and ecological data,
detectable by analyzing departures from theory. We will also discuss
the methodological advances, both in mathematics and the wet lab, that
have been achieved and will be required to achieve the most insight
from this approach \citep{gibson2014, harte2015ecolLett}.

% Lastly, we will put forward two means of
% incorporating the time perspective explicitly: 1) The fossil record provides a
% time-averaged view with deep extent but coarse resolution
% \citep{benton2000quality}.  Chronosequences provide fine resolution
% snap shots of ecological communities in varying stages of assembly but
% their use is contingent on the assumption that similar processes have
% occurred throughout the evolution of the chronosequence
% \citep{rominger2015GEB}.


\bibliographystyle{tree}
\bibliography{../mol2ecol.bib}

\end{document}



