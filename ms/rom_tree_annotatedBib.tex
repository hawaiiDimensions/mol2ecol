\documentclass[12pt]{article}
\usepackage[margin=1in]{geometry}
\geometry{letterpaper}
\usepackage{graphicx}
\usepackage{setspace}
\usepackage{amssymb}
\usepackage{amsmath}
\usepackage{epstopdf}
\usepackage[numbers]{natbib}
\usepackage{authblk}

\title{Linking evolutionary and ecological theory illuminates
  non-stationary biodiversity}

% \author[1]{A. J. Rominger}
% \author[1]{H. Krehenwinkel}
% \author[1]{R. G. Gillespie}
% \author[1]{J. Harte}
% \author[2]{M. J. Hickerson}
% \affil[1]{Department of Environmental Science, Policy and Management,
%   University of California, Berkeley}
% \affil[2]{Biology Department, City College of New York}
% \renewcommand\Affilfont{\footnotesize}

\date{}

\begin{document}
\maketitle

Equilibrial dynamics have been attempted to be constructed from
phylogenies and fossil record.  Ecological theory is based on
asymptotic stationarity.  Are these really real?  Merging community
data and theory with ecological data and theory can tell us.

\section{Stationarity}

\cite{rabosky2015amNat} argue there are carrying capacities for biodiversity

\cite{harmon2015amNat}. argue against carrying capacities for
biodiversity


\section{Next gen}

\cite{nielsen2005}
\cite{nei2010}

\section{Metabarcoding}
\citep{gibson2014}

\section{Phylogeny}

\cite{quental2010} can't infer evolutionary rates from phylogeny

\section{Ecol Theory}

\cite{McGill2017} Need to move beyond single predictions of just SAD
if we're going to test theory

\cite{Adler2005} Rates of turnover in space decrease over longer
sampling time spans, and rates of turnover in time decrease over
larger sampling areas.  Perfect symetry in space and time would
indicate stationarity and ergodicity.  The fact that SAR and STR are
not Poisson random sampling processes implies non-stationarity and
space and time.

\cite{Adler2010} Modelled plant communities assuming competitive
interactions, removed niche differences and projected invasion growth
rate and extinction time. Found excessive strength of niche
differences for given little fitness differences in need of
stabilizing mechanisms.

\cite{Adler2007} Sets up two-dimensions community space with niche
differences (stabilizing factors) and fitness differences on the two
axes.

\cite{Allen2006} show speciation rates in Neptune samples peak at
equator, explaining in part the latitudinal diversity gradient.

\cite{Allen2007} Points out that species longevities are too long
under demographic stochasticity and too short assuming a point
mutation model of speciation (also in \cite{Ricklefs2003,
  Ricklefs2006}).  Introduces environmental stochasticity and
incipient species populations $> 1$ to counteract this.  Conclude that
``we propose that a primary effect of niche differences among species
is to rescale the absolute tempo of biodiversity dynamics in an
ever-changing abiotic and biotic environment''

\cite{Alonso2004} Simplifies likelihood functions of
\cite{Etienne2004}.

\cite{Alroy1996} finds negative relationship between per genus
origination and genus richness implying diversity dependent
diversification in line with the model of Walker and Valentine
(1984). Further developed in \cite{Alroy1998}

\cite{Burbrink2015} combines phylo metrics and neutral theory analysis
to show that island herpetofauna communities are neutral.

Remember all the papers that use phylo info just to think about
traits.  We want to use info to think about historiesis.

\cite{Carroll2015} introduce fitness and niche differences into
stochastic model and show that species lifetimes is different between
neutral and non-neutral models, even if SADs are the same.

\cite{Heaney2000} explains how simplified theories (focusing on
MacWilson) are ment to be used as nulls.  Also talks about importance
of speciation on islands and possibility of a trivariate equilibrium
model (immigration, speciation, extinction).

\cite{Heaney1984} Compares SARs for oceanic islands and habitat
fragments, finds lower z-values in gragments, speaking to potential
extinction debt and non-equilibrium, finding same result as
\cite{Brown1971} that steeper z-value associated with lack of
immigration while extinction still occurs.

\cite{Brown1971} Dis-equilibrium from recent isolation of high peacks
cutting off immigration

\cite{Lowlor1986} estimates rates of species area relationships for
mammals of different dispersal ability on islands, touching on
non-equilibrium dynamics.

\cite{Wallington2005} discuss importance of non-equilibrium for
conservation thinking.

\cite{Bowler2012} confirm logseries results from statistical mechanics
in ecology.

\cite{Sax2002}, \cite{Sax2003} and \cite{Sax2007} show species
preferentially invade diverse communities, casting doubt on ideas of
niche-based competitive equilibrium of richness and assembly.

Red queen model of \cite{Odwyer2014} could
be case where genetically we see far from equilibrium, but that
non-equilibrium is the cause of rapid equilibration ecologically.

\cite{He2010} makes the point that communities are torn between
wanting to maximize entorpy and constraints imposed by ecology and
evolution

\cite{Blonder2017} talk about lagged environmental tracking as
non-equilibrium ecological dynamics.  Also has good possible citations
for conservation relevance.

\cite{Svenning2015} show evidence that past climate leaves legacy on
current diversity patterns/communities

\cite{Hoffmann2011} point to rapid adaptation as buffering populations
against extinction under climate change; conversely without
adaptation, extinciton is possible.  Both are non-equilib in either
genetic or ecological axes.

\cite{McGill2003} points out that time to LOCAL equilibrium can be up
to tens of thousands of years---potentially much longer than local
environments can be considered to be stable, implying equilibrium
neutral conditions may never be met and we're always in a neutrally
non-equilibrium world.

\cite{Hengeveld1994} rejects equilibrium outright---invokes
irriversible adaptations.

Evidence from fossil record that speciation does not increase
following extinction is good evidence against equilibrium

\cite{Tilman1994} argues for equilibirum community dynamics from
competing species in spatial context with tradeoffs in competitoin and
dispersal

\bibliographystyle{tree}
\bibliography{../mol2ecol.bib}

\end{document}



