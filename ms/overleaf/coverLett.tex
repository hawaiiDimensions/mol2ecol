\documentclass[12pt]{article}
\usepackage[margin=1in]{geometry}
\geometry{letterpaper}
\usepackage{graphicx}
\usepackage{setspace}
\usepackage{amssymb}
\usepackage{amsmath}
\usepackage{epstopdf}

% for inline enumeration env
\usepackage{paralist}

%% to make enumeration more compact
% \usepackage{enumitem}
% \setlist{noitemsep}

\usepackage[compress,numbers]{natbib}


%%%%%%%%%%%%%% begin document %%%%%%%%%%%%%%
\begin{document}
\thispagestyle{empty}

\noindent
May 08, 2017
\vspace{2em}

\noindent
To the Editors of {\it Trends in Ecology \& Evolution},
\\

Please find enclosed our manuscript ``Linking evolutionary and
ecological theory illuminates non-equilibrium biodiversity''. This
opinion piece puts forth a new analytical approach to understanding
and predicting non-equilibrium processes in ecology and evolution. We
do so by synthesizing insights gained from testing equilibrial
ecological theory as a null model, with inference into the deep
evolutionary history of communities from next generation
sequencing-enabled population/phylogenetics theory applied at the
scale of the entire community. While other researchers have proposed
(and we briefly review) syntheses between ecology and evolution using
population/phylogenetic data and/or tools, our approach represents a
fundamentally novel contribution because

\begin{enumerate}
\item It proposes a joint modeling method using next generation
  sequencing data to simultaneously inform ecological theory of the
  evolutionary history underlying a community of interest, while also
  informing evolutionary inference of what ecological outcomes are
  realistic. No attempt has been made to achieve such a quantitative
  synthesis, let alone at the scale we are proposing. This represents
  a theoretical and bioinformatic breakthrough with far-reaching
  applications from microbial ecology to massive biodiversity
  assessments and analyses of macro-organism communities
\item It provides rigorous quantitative predictions not just of basic
  patterns in ecological and genetic/genomic datasets, but also what
  these patterns reveal in terms of the systems' current state of
  equilibrium or non-equilibrium and likelihood of transitioning to
  other dynamic states
\item It solves a long-standing challenge in ecological theory, namely
  that many dynamic processes map onto the same static
  ``macroecological'' patterns, such as the species abundance
  distribution or species area relationship. Our approach does so by
  creating new predictions and dimensions of data available for theory
  testing
\item Finally our framework sets up future opportunities for
  integrating cutting edge approaches from paleontology and functional
  genomics.
\end{enumerate}

We are grateful for the chance to share our work via {\it Trends in
  Ecology \& Evolution} and hope that our framework can initiate and
invigorate new research into non-equilibrium biodiversity dynamics and
the synthesis of theories from ecology and evolution.
\\

\noindent
Sincerely,
\vspace{2em}

\noindent
Andrew J. Rominger, PhD \\
Corresponding author \\
{\tt ajrominger@gmail.com}

\end{document}

